% This is a copy version from https://github.com/thanhhungqb/thesis-template
% * <thanhhungqb@gmail.com> 2017-05-09T23:23:15.758Z:
%
% ^.
% Please do not modified this project, when you want to start writing, make a clone of it for your own (Please read README.md)

\documentclass[12pt,a4paper,oneside]{book} % twoside for draf

% \usepackage{babel}
% \usepackage[utf8]{vietnam}
\usepackage[english,spanish]{babel}
\usepackage[utf8]{inputenc}
\usepackage{float}
%\usepackage{times}
%\usepackage{graphicx}

\usepackage{mathptmx}	% same Time New Roma
%\renewcommand{\rmdefault}{phv} % Arial
%\renewcommand{\sfdefault}{phv} % Arial

\usepackage{fancyhdr}
\usepackage{array}
\usepackage{amsmath,tabu}
\usepackage{makecell}
\usepackage{url}
\usepackage{algorithm2e}
\usepackage{parskip}
\usepackage{fancyhdr}

% \usepackage{subcaption}
\usepackage{subfigure}
\usepackage{graphicx}
\usepackage{caption}
\usepackage{lipsum}
\usepackage[pdftex,pdfpagelabels,bookmarks,hyperindex,hyperfigures]{hyperref}


%prevent to break long word
\tolerance=1
\emergencystretch=\maxdimen
\hyphenpenalty=10000
\hbadness=10000

\usepackage{titlesec, blindtext, color}
\definecolor{gray75}{gray}{0.75}
\newcommand{\hsp}{\hspace{20pt}}
\titleformat{\chapter}[hang]{\Huge\bfseries}{\thechapter{.}\hsp}{0pt}{\Huge\bfseries}

\usepackage{bkthesis}

\crname{LUẬN VĂN TỐT NGHIỆP}
\ctname{XÂY DỰNG MÔ HÌNH GAN\\TỪ ẢNH CHỤP CẮT LỚP VI TÍNH CT}
\cstuname{SVTH: }

\csCouncil{Khoa học máy tính}
\csSupervise{}
\csReviewer{}
\cttime{6/2019}


\thesislayout

\begin{document}
%-	Bìa cứng - màu xanh dương, chữ mạ vàng (xem mẫu đính kèm)
%-	Trang tên (tờ lót): chất liệu giấy, nội dung giống như bìa LV
%-	Ở gáy LV: in nhan đề LV (có thể in tóm tắt nếu nhan đề quá dài), size 15 – 17
%-	Phiếu Nhiệm vụ LV, chấm điểm Hướng dẫn & Phản biện (đã ký): nhận từ GVHD & GVPB sau khi bảo vệ (theo lịch hẹn).
%-	Lời cam đoan
%-	Lời cảm ơn/ Lời ngỏ
%-	Tóm tắt LV
%-	Mục lục
%-	Danh mục, bảng biểu, hình ảnh, ... (nếu có)
%-	Nội dung LV
%-	Danh mục TL tham khảo
%-	Phụ lục (nếu có)

\coverpage

\frontmatter

\chapter*{Lời cam đoan}
\noindent 
Đưa tiến bộ khoa học kỹ thuật vào ứng dụng thực tế trong đời sống là một nhu cầu thiết yếu và luôn được 
% Khai phá dữ liệu không phải là một đề tài mới nhưng vẫn luôn là một thách thức khó khăn bởi nguồn dữ liệu đầu vào rất đa dạng, phong phú và kết quả đầu ra phải đạt được những ý nghĩa nhất định. Trong quá trình nghiên cứu đề tài có rất nhiều kiến thức không nằm trong chương trình giảng dạy ở bậc Đại học tuy vậy chúng tôi xin cam đoan đây là công trình nghiên cứu của riêng chúng tôi dưới sự hướng dẫn của tiến sĩ Phan Trọng Nhân. Nội dung nghiên cứu và các kết quả đều là trung thực và chưa từng được công bố trước đây. Các số liệu được sử dụng cho quá trình phân tích, nhận xét được chính chúng tôi thu thập từ nhiều nguồn khác nhau và sẽ được ghi rõ trong phần tài liệu tham khảo. \\


Ngoài ra, chúng tôi cũng có sử dụng một số nhận xét, đánh giá và số liệu của các tác giả khác, cơ quan tổ chức khác. Tất cả đều có trích dẫn và chú thích nguồn gốc. \\

Nếu phát hiện có bất kì sự gian lận nào, chúng tôi xin hoàn toàn chịu trách nhiệm về nội dung đề cương luận văn của mình. Trường đại học Bách Khoa thành phố Hồ Chí Minh không liên quan đến những vi phạm tác quyền, bản quyền do chúng tôi gây ra trong quá trình thực hiện.
\begin{flushright}
Sinh viên thực hiện
\end{flushright}
\chapter*{Lời cảm ơn}
\noindent 	Để hoàn thành kì đề cương luận văn này, chúng tôi tỏ lòng biết ơn sâu sắc đến tiến sĩ Phan Trọng Nhân đã hướng dẫn tận tình trong suốt quá trình nghiên cứu. \\
	
Chúng tôi chân thành cám ơn quý thầy, cô trong khoa Khoa Học Và Kỹ Thuật Máy Tính, trường Đại học Bách Khoa thành phố Hồ Chí Minh đã tận tình truyền đạt kiến thức trong những năm chúng tôi học tập ở trường. Với vốn kiến thức tích lũy được trong suốt quá trình học tập không chỉ là nền tảng cho quá trình nghiên cứu mà còn là hành trang để bước vào đời một cách tự tin. \\
	
Cuối cùng, chúng tôi xin chúc quý thầy cô dồi dào sức khỏe và thành công trong sự nghiệp cao quý.
\begin{flushright}
Sinh viên thực hiện
\end{flushright}
	
\tableofcontents
% \listofsymbols
\listoftables
\listoffigures
% \listofalgorithms


\mainmatter

\fancyhead{}  % Clears all page headers and footers
%\rhead{\thepage}  % Sets the right side header to show the page number
%\lhead{}  % Clears the left side page header
%\fancyfoot[positions]{footer}
\renewcommand{\footrulewidth}{0.4pt}

\pagestyle{fancy}  % Finally, use the "fancy" page style to implement the FancyHdr headers

\chapter{Giới thiệu đề tài}
\section{Đặt vấn đề}
Trong y khoa, trực quan hoá dữ liệu là một nhu cầu thiết thực để tăng độ chính xác trong chuẩn đoán và điều trị nhiều loại tổn thương khác nhau của bệnh nhân. Đặc biệt đối với những loại chấn thương và bệnh lý ở các cơ quan nội tạng nằm bên trong cơ thể khi mà những bộ phận này không thể được quan sát trực tiếp bằng mắt thường, việc mô phỏng lại cơ quan đó thông qua các thiết bị kỹ thuật càng trở nên thiết yếu hơn. \\
Trước đây, phương pháp chụp X-quang cho phép hiển thị được hình chiếu của các bộ phận bên trong cơ thể nhờ vào mức độ hấp thụ khác nhau của các loại mô đối với tia X. Về sau, nhờ ứng dụng nhiều công nghệ kỹ thuật tiên tiến, người ta đã có thể chụp được nhiều ảnh X-quang từ nhiều hướng khác nhau rồi từ đó sử dụng các thiết bị vi tính để tính toán ra các ảnh cắt lớp của một vùng bất kỳ trên cơ thể bệnh nhân, kỹ thuật chụp ảnh này chính là Computed Tomography (CT).\\
Mặc dù đã đạt được một bước tiến đáng kể để có thể tái tạo được những hình ảnh các cơ quan nội tạng của bệnh nhân mà không cần mổ, song muốn có được một cái nhìn tổng thể ba chiều của một cơ quan bất kỳ vẫn cần sự can thiệp lớn của các bác sĩ. Hình khối ba chiều của một cơ quan nội tạng được xây dựng từ các ảnh cắt lớp của riêng bộ phận đó. Để có được ảnh cắt lớp riêng của một bộ phận từ ảnh CT cần các bác sĩ thực hiện nhận diện và phân đoạn chính xác bộ phận đó trên từng lát cắt của ảnh. Thông thường một ảnh CT ba chiều sẽ có số lát cắt từ một trăm đến bai trăm, mộ số ảnh chi tiết hơn có thể lên tới chín trăm lát cắt. Phải tiến hành phân đoạn trên một ảnh có nhiều lát cắt là thách thức lớn nhất của các bác sĩ khi muốn trực quan hoá một cơ quan nằm bên trong cơ thể lên không gian ba chiều.\\
Để giải quyết vấn đề về phân đoạn số lát cắt ba chiều lớn, nhiều giải pháp phân đoạn tự động đã được nghiên cứu và sử dụng. Tuy nhiên độ chính xác của các phương pháp này không cao do độ tương phản của các cơ quan rất giống nhau, đặc biệt là ở lồng ngực. Các phương pháp đạt hiệu quả cao hiện tại thường là các giải pháp bán tự động và cần sự can thiệp của đôị ngũ y bác sĩ. Vì vậy trong luận văn tốt nghiệp này, chúng tôi đặt mục tiêu hoàn thiện một giải pháp có tính ứng dụng thực tế để chuyển ảnh chụp CT gồm nhiều lát cắt thành một mô hình 3-chiều, và cơ quan nội tạng được chúng tôi tập trung nghiên cứu phân đoạn ở đây là lá gan.
\section{Mục tiêu đề tài}
Trong luận văn tốt nghiệp này, chúng tôi sẽ xây dựng một giải pháp phân đoạn tự động trên cơ sở của mạng học sâu để có thể trích xuất được thông tin bề mặt và trực quan hoá ba chiều của lá gan từ ảnh CT lồng ngực. Mô hình của chúng tôi sẽ được tiến hành so sánh với các mô hình khác đã được các Luận văn tốt nghiệp trước trình bày. Bên cạnh đó chúng tôi cũng sẽ sử dụng các công cụ và tập dữ liệu đã được công khai có tính khách quan cao để đánh giá kết quả của mô hình được phát triển trong đề tài này. Và cuối cùng, chúng tôi sẽ hiện thực một ứng dụng cho phép phân đoạn tự động lá gan từ ảnh CT được người dùng đưa vào và hiển thị mô hình ba chiều cùng với ước lượng thể tích của lá gan đó.

\begin{figure*}
  \centering
  \subfigure[Ảnh CT lồng ngực]{%
    \includegraphics[width=0.3\textwidth]{Images/image5.pdf}%
    \label{fig:image5}%
    }
    \subfigure[Ảnh phân đoạn lá gan]{%
    \includegraphics[width=0.3\textwidth]{Images/label5.pdf}%
    \label{fig:b}%
    }
    \subfigure[Ảnh lá gan 3 chiều]{%
    \includegraphics[width=0.3\textwidth]{Images/liver3d.png}%
    \label{fig:c}%
    }% 
  \caption{Minh hoạ quá trình xây dựng lá gan 3 chiều từ ảnh CT lồng ngực}
  \label{fig:ab}
\end{figure*}


\section{Phạm vi thực hiện}
Trong khuôn khổ của một Luận văn tốt nghiệp, chúng tôi sẽ giới hạn lại phạm vi nghiên cứu phù hợp để có thể đảm bảo được tính ứng dụng của đề tài và tính khách quan khi đánh giá mô hình được xây dựng với các mô hình khác. Ở đề tài này, chúng tôi sẽ tập trung vào xây dựng mô hình lá gan từ ảnh CT chụp lồng ngực bằng phương pháp chính là học sâu. Mô hình của chúng tôi sẽ được đánh giá với mô hình hai mô hình của anh Đàm Vũ Duy và của nhóm hai anh Bùi Hồng Thiên Nhật - Phạm Huỳnh Sơn. Giải pháp của chúng tôi cũng sẽ được đánh giá với các tập dữ liệu đã được sử dụng tại các hội nghị về xử lý ảnh y khoa có uy tín cáo SLiver07 \cite{website:slvier07}, 3Dircadb \cite{website:data_3DIRCADb}, LiTS2017 \cite{website:LiTS}.
\chapter{Trình tự công việc và bố cục luận văn}
\section{Trình tự công việc}
Để đạt được mục tiêu đề tài đặt ra, chúng tôi lần lượt tiến hành những công việc sau
\begin{enumerate}
    \item Khảo sát các tập dữ liệu sử dụng.
    \item Khảo sát các phương pháp xử lý ảnh giành cho kiểu dữ liệu vừa khảo sát.
    \item Khảo sát phương án và kết quả khi ứng dụng mạng học sâu vào xử lý ảnh y khoa.
    \item Tập trung nghiên cứu các mô hình học sâu có thể ứng dụng cho đề tài.
    \item Tiến hành xây dựng và thử nghiệm các mô hình học sâu trên các tập dữ liệu được công khai.
    \item Áp dụng một số kỹ thuật hậu xử lý phân đoạn để được kết quả chính xác hơn.
    \item Đánh giá các kết quả trên hệ thống Sliver07 và với phương pháp trong các luận văn trước.
    \item Đóng gói mô hình và xây dựng ứng dụng hiển thị.
\end{enumerate}
\section{Bố cục luận văn}
Luận văn này được trình bày thành 10 mục sắp xếp dựa trên trình tự thực hiện công việc của chúng tôi gồm:
\begin{enumerate}
    \item Giới thiệu đề tài
    \item Trình tự công việc và bố cục báo cáo
    \item Cơ sở lý thuyết
    \item Các mô hình mạng tham khảo chính
    \item Khảo sát dữ liệu và tiền xử lý dữ liệu
    \item Hậu xử lý kết quả
    \item Thử nghiệm mô hình
    \item Tổng hợp, so sánh kết quả hệ thống phân đoạn với những phương pháp khác
    \item Xây dựng ứng dụng
    \item Tổng kết
\end{enumerate}
\chapter{Cơ sở lý thuyết}
\section{Ảnh kỹ thuật số}
Ảnh kỹ thuật số là ảnh ở định dạng có thể lưu trữ trong và truyền tải qua các thiết bị điện tử. Hiện tại có hai loại định dạng cho ảnh kỹ thuật số là raster và vector. Ảnh raster được tạo thành từ các điểm ảnh được gọi là pixel. Ảnh vector được biểu diễn bằng các phép toán để tạo ra các đường nét trong ảnh. Trong luận văn này khi đề cập đến khái niệm ảnh hoặc ảnh kỹ thuật số nếu không nói gì thêm thì mặc định đó là ảnh raster.\\
Với ảnh kỹ thuật số, mỗi điểm ảnh sẽ được lưu bằng một hoặc nhiều giá trị nguyên, số giá trị nguyên đó được gọi là số kênh. Tuỳ vào từng định dạng ảnh khác nhau mà số kếnh và khoảng giá trị của mỗi kênh sẽ khác nhau.\\
Các ảnh CT y khoa tuân theo chuẩn DICOM (Digital Image and Communications in Medicine) thông thường sẽ có một kênh và được biểu thị bằng số nguyên 16-bit và được tổ chức xếp nhiều ảnh với nhau thành ảnh 3-chiều. Lúc này điểm ảnh pixel trên không gian hai chiều sẽ trở thành điểm ảnh voxel trên không gian ba chiều. Ảnh CT 3-chiều này sẽ đi kèm theo các thông tin về khoảng cách thực tế giữa hai voxel gần nhau trong không gian ba chiều. Nhờ đó ta có thể tính được diện tích và thể tích của vật thể nhờ vào số voxel của ảnh đó.

\section{Phân đoạn ảnh và các kỹ thuật phân đoạn cơ bản}
Trong bộ môn Thị giác máy tính, khái niệm phân đoạn ảnh được dùng để chỉ quá trình phân tách ảnh thành những tập hợp các pixel. Mục đích của quá trình này để giản lược hoặc thay đổi cách hiển thị ảnh có ý nghĩa và dễ để phân tích hơn. Một cách hiểu khác, phân đoạn ảnh là quá trình đánh nhãn cho từng pixel ảnh sao cho những pixel có cùng nhãn sẽ có cùng một số tính chất.\\
Khi ta thực hiện phân đoạn ảnh trên một chồng các ảnh của cùng vật thể theo đúng thứ tự, ví dụ như ảnh CT, kết quả của các phép phân đoạn này có thể dùng để tạo ra hình khối 3-chiều với các giải thuật như Marching cubes.\\
Một số kỹ thuật phân đoạn ảnh cơ bản có thể kể đến như:
\begin{itemize}
\item Phân đoạn ảnh theo ngưỡng: Phương pháp này khá đơn giản, nó chủ yếu dựa vào mức độ xám của các vùng trên ảnh để phân đoạn. Giải thuật thường được sử dụng là phương sai giữa các lớp lớn nhất(Otsu).
\item Phân đoạn dựa trên miền đồng nhất: Các miền trong ảnh sẽ dựa vào các tính chất (các tính chất sẽ có những giá trị khoảng về mức xám, màu sắc...) để phân đoạn. Từ ý tưởng này thì chúng ta có các cách sử dụng cho phương pháp này như tách cây tứ phân, cục bộ, tổng hợp…
\item Phân đoạn dựa vào biên của các đối tượng: Dựa trên một số giải thuật phát hiện biên như Sobel, Laplacian,... Những giải thuật này sẽ tìm được biên của một số đối tượng dựa trên các đặc điểm về chênh lệch màu sắc, mức xám trên ảnh. Từ đó chúng ta có thể phân đoạn được ảnh theo biên.
\end{itemize}
\textbf{Nhận xét:} Những phương pháp này có thể gọi là những phương pháp cổ điển để phân đoạn ảnh. Đặc điểm chung của nó là việc thực hiện nhanh chỉ qua vài phép toán biến đổi. Còn nhược điểm là bị sai nhiều khi các đối tượng trên ảnh không có sự khác nhau rõ ràng về màu sắc, mức xám. Việc sử dụng các phương pháp này chỉ để nghiên cứu và đánh giá hoặc công cụ hỗ trợ còn để ứng dụng vào thực tế thì vẫn rất hạn chế vì với các bài toán phức tạp thì các phương pháp này không thể đáp ứng được. Chúng ta có thể tìm hiểu thêm về các phương pháp này tại đây.\cite{segoverview}\\
Phương pháp tốt nhất hiện nay áp dụng cho các bài toán phân đoạn ảnh phức tạp là mạng học sâu. Tuy rằng chi phí để hiện thực và chạy mạng học sâu rất cao nhưng với các công cụ hỗ trợ hiện nay kết hợp với việc tận dụng được hiệu năng của GPU của mạng học sâu thì ngày nay nó được sử dụng rất nhiều và mang lại những kết quả vượt ngoài sức mong đợi. Và để giải thích cho điều này thì bên dưới sẽ trình bày chi tiết về mạng học sâu bao gồm cơ sở lý thuyết, trình bày và đánh giá một số mô hình trong mạng học sâu và một số kết quả mà nhóm đã làm được khi sử dụng nó.

\section{Mạng nơ-ron học sâu}
Phần này chúng tôi sẽ trình bày lại quá trình phát triển của mạng nơ-ron học sâu và phần nào lý giải nguyên do cho sự bùng nổ ứng dụng của mạng nơ-ron trong thời điểm hiện tại. Nội dung được chúng tôi tham khảo tại \cite{basicdeep}.
\subsection{Một số khái niệm cơ bản}
Trước hết chúng tôi nhắc lại các khái niệm như trí tuệ nhân tạo(Artificial Intelligence), học máy (Machine Learning) và học sâu(Deep Learning) viết tắt lần lượt là AI, ML và DL.\\
Khi máy tính có khả năng thực hiện những việc mà có đặc trưng như trí thông minh con người thì ta có thể gọi đó là trí tuệ nhân tạo. Khái niệm học máy được sử dụng khi máy tính có khả năng học từ dữ liệu được lưu trữ mà không phải lập trình một cách cụ thể. Học sâu là một trong những phương pháp của học máy sử dụng các mô hình mạng nơ-ron, ý tưởng chính dựa các nơ-ron và kết nối giữa chúng trong bộ não sinh học. Ta minh họa ba khái niệm này trong Hình \ref{history}.

\begin{figure}[ht]
\centering
        \includegraphics[totalheight=8cm]{Images/history.png}
    \caption{Minh họa AI, ML và DL\cite{basicai}}
    \label{history}
\end{figure}

\subsection{Lịch sử deep learning và những kiến thức cơ bản về nó}
\textbf{Giải thuật Perceptron Learning Algorithm(PLA)}
\begin{itemize}
\item PLA là nền móng đầu tiên của DL, nó được ra đời vào năm 1957 bởi Frank Rosenblatt giải bài toán về phân lớp nhị phân. Vì nó là nền móng nên hiểu được cách hoạt động của giải thuật này sẽ hiểu được cơ bản về DL. Vì sau này khi phát triển lên người ta chỉ cả tiến nó chứ không phải thay thế.
\item Yêu cầu của bài toán đặt ra là tìm phương trình một đường thẳng phân hai lớp đã được gán nhãn nằm về hai phía của đường thẳng đó, gọi là class 1 và class 2. Ý tưởng cơ bản của PLA là dự đoán, xây hàm mất mát, dựa vào hàm mất mát để đánh giá dự đoán và cập nhật lại dự đoán mới khi nào tốt thì dừng lại.
\item Trong không gian n chiều gọi \textbf{x} = [$x_{1}$, $x_{2}$, ..., $x_{n}$] là tập hợp các điểm dữ liệu, mỗi điểm được biểu diễn bằng một vector n chiều. \textbf{y} = [$y_{1}$, $y_{2}$, ..., $y_{n}$] là nhãn tức là giá trị lớp của mỗi điểm trong \textbf{x} tương ứng, $y_{i}$ = 1 nếu $x_{i}$ thuộc class 1, $y_{i}$ = -1 nếu $x_{i}$ thuộc class 2.
\item Ta giả sử đường thẳng cần tìm có phương trình:\[f_{x}(x) = w_{1}x_{1}+...+w_{n}x_{n}+w_{0}=w^{T}x = 0\]
Công việc của chúng ta là đi tìm các giá trị $w_{i}$ này hay còn gọi là các trọng số (weights).
\item Để dễ hình dung khi đi vào tính toán, ta sẽ đi tiếp trong trường hợp là không gian hai chiều (các công thức vẫn viết tổng quát cho n chiều). Ta giả sử đường thẳng $w_{1}$$x_{1}$ + $w_{2}$$x_{2}$ + $w_{0}$ = 0 là nghiệm của bài toán. Khi đó với điểm x chưa có nhãn, ta có thể gán nhãn cho nó bằng cách  \[label(x) = 1 \quad if \quad w^T \geq 0, \quad else \quad -1\] hay ngắn gọn là label(x) = sgn($w^{T}x$) với sgn là hàm xác định dấu.
\item Hàm mất mát này tính dựa trên số điểm bị phân lớp sai. Với mỗi điểm phân lớp sai thì y và $w^{T}$$x_{i}$ trái dấu, và khi phân lớp sai khoảng cách của điểm đó càng lớn thì giá trị hàm lỗi càng tăng. Việc chúng ta cần làm là đi tối ưu hàm lỗi này từ đó cập nhật được trọng số $w$. Ở đây sẽ đề xuất một phương pháp tối ưu áp dụng Gradient Descent đó là xét mỗi điểm bị phân lớp sai, khi đó giá trị mất mát cho điểm đó là: \[J(w; x_i; y_i) = -y_i w^T x_i\] có đạo hàm là $\Delta _{w}$J(w; $x_{i}$; $y_{i}$) = -$y_{i}$$x_{i}$.
\item Ta có quy tắc cập nhật là: $w_{t+1}^{T}$$x_{i}$ = ($w_{t}+y_{i} x_{i} )^{T}$$x_{i}$ =  $w_{t}^{T}$$x_{i}$ + $y_{i}$ $\left \| x_i \right \| ^{2}$.
Nhận thấy khi $w_{t}^T$$x_{i}$ và $y_{i}$ trái dấu, khi cập nhật lại thì $w_{t+1}^Tx_{i}$ < $w_{t}x_{i}$ nên hàm lỗi sẽ nhỏ đi theo mỗi bước cập nhật. Kết quả cuối cùng sẽ tối ưu nếu tập input có dạng tuyến tính. Mô hình minh họa cho PLA như Hình \ref{pla}. 
\end{itemize}

\begin{figure}[ht]
\centering
        \includegraphics[totalheight=5cm]{Images/first_ann.png}
    \caption{Mô hình nơ-ron cho PLA \cite{basicdeep}}
    \label{pla}
\end{figure}

\begin{itemize}
\item Qua đây ta có thể giải thích được một số thành phần cơ bản trong một mạng nơ-ron. Lớp input là các giá trị $x$, các giá trị $w$ là các trọng số(weights), giá trị $x_{0}$=1 sau này sẽ được gọi là bias.
\item Với $z = w^{T}x$, ta có $y=sgn(z)$ là output của mạng. Hàm $sgn()$ được gọi là activation function.
Còn hàm lỗi thì chúng ta tự xây dựng và tự tối ưu miễn sao giải thuật có hiệu quả.
\end{itemize}
\textbf{MLP và Backpropagaion}
\begin{itemize}
\item Mặc dù PLA mang lại nhiều triển vọng lúc mới ra đời nhưng về bản chất thì nó không thể giải quyết được với tập dữ liệu phi tuyến và nó được chứng minh trong cuốn sách Perceptrons, điều này làm cho giải thuật này không phát triển được thêm trong hai mươi năm. Thời kỳ này còn được gọi là mùa đông AI thứ nhất(The first AI winter).
\item Mãi tới năm 1986, Geoffrey Hinton cùng cộng sự viết một bài báo có tên Learning representations by back propagating error chứng minh được rằng mạng nơ-ron với nhiều lớp ở giữa(gọi là hidden layer) có thể biểu diễn được các quan hệ phi tuyến với quy trình gọi là backpropagation và sau mỗi lớp là một hàm kích hoạt phi tuyến sigmoid hoặc tanh. Lúc này nó được gọi là Multi-layer Perceptron(MLP).
\item MLP cải tiến thêm chỗ thay vì chỉ có hai lớp input và output như PLA thì nó sẽ có thêm nhiều hidden layer ở giữa. Mỗi node ở các lớp này gọi là một Unit, input của mỗi Unit sẽ là output của lớp trước đó qua activation function. Giá trị $z$ ở mỗi Unit sẽ cộng thêm một giá trị $b$ được gọi là bias. Với MLP thì giá trị $b$ này không cố định là $1$ mà nó sẽ thay đổi trong quá trình học, qua các lớp, trên từng Unit.
\item Backpropagation là một phương pháp sử dụng để tối ưu hàm lỗi cho MLP, nó giúp tính gradient ngược từ lớp cuối cùng đến lớp đầu tiên dựa vào cách tính đạo hàm của hàm hợp. Nhờ đây mà mạng nơ-ron tiếp tục được phát triển cho tới khoảng 1990 do gặp phải vấn đề tiếp theo.
\end{itemize}
\textbf{Mùa đông AI thứ hai và sự ra đời của mạng học sâu}
\begin{itemize}
\item Vấn đề phi tuyến đã được giải quyết nhưng việc tối ưu hàm lỗi của mạng nơ-ron quá phức tạp khi gặp bài toán khó và dữ liệu đầu vào lớn vì nó không phải là một hàm lồi. Tiếp đến là việc các hidden layer mở rộng làm cho khả năng tính toán mất quá nhiều chi phí, làm cho việc huấn luyện không hiệu quả, và hơn nữa là vấn đề vanishing gradient (gradient bỏ qua những thành phần có đạo hàm =0). Những hạn chế này làm cho mạng nơ-ron bị thay thế dần bởi một giải thuật mới có tên là Support Vector Machine(SVM) trong giai đoạn 1990-2000. Giai đoạn này còn gọi là Mùa đông AI thứ hai.
\item Tới 2006, mạng nơ-ron được cải tiến bởi Hinton với khái niệm autoencoder. Ta có thể hiểu khái niệm nay bằng cách xây dựng nó như sau. Xây dựng một autoencoder là xây dựng một mạng nơ-ron chỉ có một hidden layer, số Unit của hidden layer này ít hơn số input và output thì bằng với input. Ta sẽ huấn luyện mạng này sao cho giá trị output giống với giá trị input. Vì mạng chỉ có một hidden layer nên ta có thể gọi giai đoạn từ input tới hidden layer là mã hóa, còn từ hidden layer tới output là giải mã. Quá trình này thành công, ta sẽ bỏ đi output vì về bản chất thì hidden layer tuy ít node hơn nhưng vẫn mang đầy đủ tính chất của input. Ta tiếp tục xây dựng một autoencoder từ hidden layer này và cứ thế cho đến khi đạt được một mạng đủ sâu mà output của mạng này mang được những đặc điểm của input. Sau đó ta tiếp tục làm việc với output này theo yêu cầu bài toán. Việc này sẽ làm giảm thiếu rất nhiều vấn đề vanishing gradient. Từ ý tưởng này, mạng nơ-ron được cộng đồng tiếp tục đón nhận và cũng kể từ đây cái tên mạng học sâu(Deep Learning) ra đời.
\item Năm 2012 là năm có nhiều biến động nhất, cũng kể từ đây mà DL chiếm ưu thế vượt trội hoàn toàn so với các phương pháp khác trong cùng lĩnh vực. Đầu tiên là việc ra đời của ReLU và dropout do Geoffrey Hinton và cộng sự công bố, hai kỹ thuật này khá đơn giản nhưng lại đem lại kết quả ngoài sức mong đợi. ReLU là hàm với output là 1 khi đầu vào không âm, là 0 khi ngược lại, còn dropout là trong quá trình huấn luyện một số hidden unit bị tắt ngẫu nhiên và khi test thì sẽ kích hoạt lại toàn bộ. Những phương pháp này mang lại nhiều hiệu quả và giảm thiểu được vanishing gradient. Tiếp đến là việc tận dụng được hiệu năng của GPU trong quá trình huấn luyện (GPU là card đồ họa với mục đích chính sản xuất là để chơi game với khả năng chạy song song nhiều lõi nhưng lại rất thích hợp khi huấn luyện các mạng học sâu, làm cho việc huấn luyện nhanh hơn rất nhiều lần).
\item Từ đó tới nay, mạng học sâu được sử dụng một cách rộng rãi hơn, các sản phẩm sử dụng nó, các nhà nghiên cứu, người học đều tăng cao, đồng thời các công cụ hỗ trợ việc code và huấn luyện cũng tăng lên nhanh chóng.
\end{itemize}

\section{Mạng nơ-ron học sâu tích chập và ứng dụng cho phân đoạn ảnh}
Các mạng nơ-ron truyền thống có thể giải những bài toán phức tạp với độ chính xác khá cao nhưng vấn đề chi phí đôi khi lại làm cho nó trở nên không khả thi. Đặc biệt sử dụng nó trong lĩnh vực xử lý ảnh. Ta lấy ví dụ một ảnh đầu vào có kích thước dài 128, rộng là 128 , ba giá trị màu là RGB sẽ có tới 128x128x3 = 49125 tham số cho mỗi lớp mạng để xử lý bức ảnh này. Hơn nữa các mạng nơ-ron phức tạp còn mắc phải vấn đề overfitting. Để giải quyết vấn đề này thì ta có khái niệm mạng nơ-ron tích chập (Convolutional Neural Networks hay ngắn gọn là CNN) là một dạng cải tiến của mạng nơ-ron nhân tạo. Ta sẽ đi vào phân tích một số đặc điểm của CNN để thấy rõ điều này.

% Toàn bộ kiến trúc CNN được minh họa như Hình \ref{cnn}.

% \begin{figure}[ht]
% \centering
%         \includegraphics[totalheight=9cm]{Images/cnn.png}
%     \caption{Kiến trúc CNN\cite{accarchitect}}
%     \label{cnn}
% \end{figure}

\subsection{Lớp tích chập}
Trong xử lý ảnh và thị giác máy tính, người ta rõ hai khái niệm phép toán tích chập (convolution) và phép toán tương quan (correlation) qua điểm khác nhau duy nhất ở quá trình tính toán là đảo ngược ma trận lõi ở phép tích chập trước khi tính. Để dễ hình dung ta xem xét ví dụ sau đây với giá trị $I$ (độ dài $i$) và $K$ (đọ dài $k$):
$$ I = [1, 2, 3, 4, 5, 6, 7, 8, 9]$$
$$ K = [2, 4, 6] $$
Thì kết quả của phép tính tương quan được tính như sau:
$$ Correlation(I,K)_{c = \overline{0, i - k - 1}} = \sum_{m=c}^{c + k-1}I_{m}*K_{m-c}$$
$$ Correlation(I,K) = [1*2 + 2*4 + 3*6, ... , 7*2 + 8*4 + 9*6] $$
$$ Correlation(I,K) = [28,  40,  52,  64,  76,  88, 100] $$
Và kết quả của phép tính tích chập được tính như sau:
$$ Convolution(I,K)_{c = \overline{0, i - k - 1}} = \sum_{m=c}^{c + k-1}I_{m}*K_{k - 1 - m + c}$$
$$ Convolution(I,K) = [1*6 + 2*4 + 3*2, ... , 7*6 + 8*4 + 9*2] $$
$$ Convolution(I,K) = [20, 32, 44, 56, 68, 80, 92] $$
Công thức tương tự khi I, K là các ma trận 2, 3 chiều. Để dễ hình dung ta xem hình minh hoạ \ref{convExample}.

\begin{figure}[ht]
\centering
        \includegraphics[totalheight=7cm]{Images/convExample.png}
    \caption{Phép tính tương quan trong xử lý ảnh}
    \label{convExample}
\end{figure}
Ta có thể thấy ở phép tính tương quan trên ma trận, tuỳ vào ma trận nhân $K$ mà ma trận kết quả sẽ có được một số tính chất đặt biệt, ví dụ ở hình \ref{convExample} ta sẽ nhận được ma trận kết quả làm nổi bật các cạnh dọc ở ma trận gốc. Điều tương tự cũng xảy ra với kết quả của phép tính tích chập vì hai phép tính này là gần giống nhau ở cách tính. Trong các phép tính xử lý ảnh, tuỳ vào tình huống mà người ta sử dụng phép tích chập hoặc phép tính tương quan để nhầm đạt được độ lợi về thời gian tính toán khi tính áp dụng các tính chất kết hợp lên một dãy các phép tính tích chập (tương quan) liên tục\cite{convtheorem}.\\
Tuy nhiên ở các mạng học sâu, lớp tích chập và phép tính tích chập được dùng để chỉ phép tính tương quan. Các nên tảng học sâu được hỗ trợ như Tensorflow sử dụng từ khoá tích chập và lớp tích chập để chỉ phép tính tương quan trong xử lý ảnh. Vì như đã nói ý nghĩa của hai phép tính này là như nhau, tuỳ vào độ lợi tính toán mà người ta sẽ quyết định sử dụng phép tính nào.\\
Khi sử dụng lớp tích chập thay cho lớp kết nối đầy đủ ở mạng học sâu sẽ có một số khác biệt sau:
\begin{itemize}
    \item Bộ nhớ tính toán sử dụng các lớp tích chập sẽ nhỏ hơn.
    \item Thời gian tính toán khi sử dụng các lớp tích chập sẽ nhanh hơn.
    \item Ở những lớp càng sâu, một điểm dữ liệu khi sử dụng các lớp tích chập sẽ bị ảnh hưởng bới nhiều điểm dữ liệu đầu vào hơn thay vì là như nhau ở tất cả các lớp.
\end{itemize}

\subsection{Lớp gộp}
Lớp gộp (pooling), đôi khi được gọi là lớp giảm kích thước mẫu (sub-sampling), thường được sử dụng sau một số lớp tích chập nhất định. Ta có thể thấy sau khi qua một số lớp tích chập nhất định, ảnh thường sẽ có hiện tượng bị nhoè do những điểm ảnh gần nhau có giá trị gần giống nhau. Lớp gộp này đóng vai trò làm giảm kích thước ảnh nhưng vẫn đảm bảo giữ lại các dữ liệu cần thiết nhằm qua đó làm giảm số lượng phép tính toán và tránh tình trạng quá khớp (overfitting).\\
Lớp gộp có thể sử dụng nhiều loại hàm gộp khác nhau như hàm trung bình hoặc hàm lấy lớn nhất (phổ biến) tuỳ thuộc vào đặc điểm của bài toán. Các lớp gộp này thường có kích thước bằng với độ cách (stride), tức là chúng sẽ có ý nghĩa như một phép thay thế một vùng các điểm dữ liệu chỉ bằng một điểm dữ liệu duy nhất, và các vùng này không có điểm trùng.\\
Ta có thể thấy sau khi qua một lớp gộp, các vùng khác nhau sẽ nằm gần nhau hơn và nhờ vậy với lớp tính tích chập tiếp theo sẽ có được thông tin của một vùng lớn hơn ở ảnh ban đầu. Vì vậy sau khi qua càng nhiều lớp gộp, ảnh (đặc trưng) sẽ có ý nghĩa toàn cục hơn là ý nghĩa chi tiết. Đây cũng là một yếu điểm của lớp gộp cho bài toán phân đoạn ảnh mà không thay đổi kích thước của nó.

\begin{figure}[ht]
\centering
        \includegraphics[totalheight=4cm]{Images/MaxpoolSample2.png}
    \caption{Phép tính gộp lấy lớn nhất với kích thước 2, độ cách 2}
    \label{poolExample}
\end{figure}

\subsection{Lớp đảo tích chập}

\section{Các chỉ số đánh giá kết quả phân đoạn}
Kết quả phân đoạn có thể được đánh giá bằng nhiều cách khác nhau. Đối với một bài toán nhất định, mỗi chỉ số đánh giá được lựa chọn cần phải nhạy cảm với một số đặc điểm phụ thuộc vào mục đích bài toán đó. Nếu bài toán để xác định thể tích vật thể, sai khác thể tích sẽ là chỉ số tiên quyết cho bài toán đó. Tuy nhiên với một kết quả phân đoạn cho ra thể tích chính xác hoàn toàn trên lý thuyết vẫn có thể bị sai hoàn toàn nếu như sử dụng chỉ số trùng giao trên hợp của voxel. Vì vậy, mỗi bài toán phải luôn dùng nhiều chỉ số phân đoạn khác nhau để đánh giá các giải pháp cho bài toán đó.\\
Trong luận văn này, chúng tôi sử dụng các chỉ số của cuộc thi phân đoạn lá gan trên tập dữ liệu SLiver07 tham khảo tại \cite{SLiver07metric}. Các chỉ số này sẽ được tính ra điểm cho từng chỉ số và cuối cùng tính trung bình ra tổng điểm cho một kết quả phân đoạn. Với mỗi chỉ số, mức điểm hoàn hảo là 100 điểm, mức điểm chấp nhận được là 75 điểm, mức điểm thấp nhất là 0 điểm. Mức điểm hoàn hảo là kết quả phân đoạn của chuyên gia đã có kinh nghiệm cho lĩnh vực phân đoạn bộ phận đó. Mức điểm chấp nhận được là của sinh viên y khoa chưa có kinh nghiệm phân đoạn khi so với kết quả của chuyên gia.\\
Các chỉ số được chúng tôi sử dụng bao gồm:
\begin{itemize}
    \item Chỉ số lỗi thể tích (Volumetric overlap error - VOE), giá trị phần trăm. Chỉ số này được tính như sau: lấy tổng số voxel giao giữa dự đoán và nhãn chia cho tổng số voxel hợp giữa dự đoán và nhãn, lấy 1 trừ kết quả đó rồi nhân cho 100.
    \item Chỉ số sai khác thể tích tuyệt đối (Relative absolute volume difference - VD), giá trị phần trăm. Chỉ số này được tính bằng hiệu số thể tích của phép phân đoạn và nhãn, chia cho nhãn, nhân 100. Chỉ số này nhận giá trị âm khi thể tích phân đoạn nhỏ hơn thể tích nhãn và nhận giá trị dương khi thể tích phân đoạn lớn hơn thể tích nhãn. Lưu ý chỉ số này có thể đạt điểm 100 khi kết quả phân đoạn không trùng với nhãn.
    \item Khoảng cách mặt đối xứng trung bình (Average symmetric surface distance - AvgD), giá trị milimet. Voxel cạnh của vật thể là voxel có ít nhất 1 trong 18 láng giềng không thuộc vật thể. Khoảng cách mỗi voxel tới bề mặt là khoảng cách ngắn nhất từ voxel đó đến một voxel thuộc bề mặt. Khoảng cách trung bình giữa hai bề mặt là giá trị trung bình của tất cả các voxel tới bề mặt kia (voxel thuộc dự đoán tới bề mặt nhãn, voxel thuộc nhãn tới bề mặt dự đoán).
    \item Khoảng cách mặt đối xứng bình phương từ gốc (Root Mean Square symmetric surface distance - RMSD), giá trị milimet. Giá trị này được tính giống với  AvgD nhưng khoảng cách ban đầu sẽ được bình phương lên và lấy trung bình.
    \item Khoảng cách mặt lớn nhất (Maximum symmetric surface distance - MaxD), giá trị milimet. Chỉ số này tính giống AD nhưng thay vì lưu trung bình sẽ lưu giá trị lớn nhất.
\end{itemize}
Lưu ý các chỉ số trên là các chỉ số lỗi, càng thấp tức là kết quả phân đoạn càng tốt. Điểm cho mỗi chỉ số sẽ bằng 100 trừ đi chỉ số nhân với một hệ số đã xác định, và lấy lớn hơn với 0. \\

\begin{table}
\begin{center}
\begin{tabular}{ |l|c| } 
\hline
\Gape[0.1cm][0.1cm]{Tên chỉ số} & Công thức chỉ số \\ 
\hline
\textbf{VOE} &\Gape[0.5cm][0.5cm]{$(1 - {R \bigcap G}/{R\bigcup G})*100\%$ }\\ 
\hline
\textbf{VD} & \Gape[0.5cm][0.5cm]{$(|R| -|G|)/|G|*100\%$} \\ 
\hline
\textbf{AvgD} & \Gape[0.5cm][0.5cm]{$(\sum d(S(G),S(R)) + \sum d(S(R),S(G)))/(|S| + |G|)$}\\
\hline
\textbf{RMSD} & \Gape[0.5cm][0.5cm]{$\sqrt{(\sum d(S(G),S(R))^{2} + \sum d(S(R),S(G))^{2})/(|S| + |G|)}$} \\ 
\hline
\textbf{MaxD} & \Gape[0.5cm][0.5cm]{$max(max(d(S(G),S(R))), max(d(S(G),S(R))))$} \\ 
\hline
\end{tabular}
\caption{\label{tab:table-name1}Cách tính các chỉ số.}
\end{center}
\end{table}


\begin{table}[H]
\begin{center}
\begin{tabular}{ |l|c| } 
\hline
\Gape[0.1cm][0.1cm]{Tên chỉ số} & Công thức điểm \\ 
\hline
\textbf{VOE} & \Gape[0.5cm][0.5cm]{$100 - 100*\textbf{VOE}*25/6.4$} \\ 
\hline
\textbf{VD} & \Gape[0.5cm][0.5cm]{$100 - 100*\textbf{|VD|}*25/4.7$} \\ 
\hline
\textbf{AvgD} & \Gape[0.5cm][0.5cm]{$100 - 100*\textbf{AvgD}*25$} \\
\hline
\textbf{RMSD} & \Gape[0.5cm][0.5cm]{$100 - 100*\textbf{RMSD}*25/1.8$} \\ 
\hline
\textbf{MaxD} &\Gape[0.5cm][0.5cm]{ $100 - 100*\textbf{MaxD}*25/4.7$} \\ 
\hline
\end{tabular}
\caption{\label{tab:table-name}Cách tính điểm cho các chỉ số.}
\end{center}
\end{table}
 
Ý nghĩa của các ký hiệu như sau:
\begin{itemize}
    \item $G$: vật thể nhãn
    \item $R$: vật thể được dựa đoán bởi mô hình
    \item $\bigcap$: phần thể tích giao của hai vật thể
    \item $\bigcup$: phần thể tích hợp của hai vật thể
    \item $|R|$: tổng số voxel của vật thể R
    \item $S(R)$: bề mặt vật thể R
    \item $d(x,S(R))$: khoảng cách từ voxel x đến bề mặt vật thể R
    \item $d(S(G),S(R))$: khoảng cách một chiều từ bề mặt vật thể G tới bề mặt vật thể R, đây là tập hợp $\{ d(v_{0}, S(R)), d(v_{1}, S(R)),...d(v_{n}, S(R)) \} $ với $v \in S(G) $
    \item $\textbf{|VD|}$: giá trị tuyệt đối của chỉ số VD.
\end{itemize}
\chapter{Các mô hình tham khảo chính}
\section{Định hướng phát triển mô hình}
\section{Mô hình đảo tích chập}
\section{Mô hình Unet}


\chapter{Khảo sát dữ liệu và tiền xử lý dữ liệu}
\section{Các tập dữ liệu sử dụng và đặc điểm}
\section{Tiền xử lý dữ liệu}



\chapter{Hậu xử lý kết quả}
Dữ liệu ảnh 3-chiều sau khi được dự đoán bởi mạng học sâu sẽ là một mảng 3-chiều số thực có giá trị trong đoạn $[0,1]$. Mảng này sẽ được chúng tôi tiếp tục xử lý để ra kết quả cuối cùng có giá trị nhất. Để giảm thiểu thời gian tính toán và đúng với tinh thần ứng dụng mạng học sâu cho phân đoạn ảnh, các bước hậu xử lý của chúng tôi chỉ sử dụng các thao tác đơn giản gồm: loại bỏ các khối nhiễu, lấp đầy các lỗ trống trong khối.
\section{Đánh nhãn vùng liên thông}
Đánh nhãn vùng liên thông (Connected-component labeling) là tác vụ cơ sở để chúng tôi thực hiện hai thao tác hậu xử lý là loại bỏ khối và lấp đầy khối. Tác vụ đánh nhãn này thường được sử dụng để xử lý các ảnh nhị phân cho các ứng dụng trong lĩnh vực thị giác máy tính. Tuy nhiên ảnh màu và ảnh có số chiều nhiều hơn hai vẫn có thể sử dụng giải pháp này.\\
Các giải thuật đánh nhãn vùng liên thông coi ảnh đầu vào là một đồ thị, điểm ảnh là đỉnh, mỗi điểm ảnh có cạnh với các điểm lân cận của điểm đó. Tuỳ vào bài toán mà người ta quy định các cạnh của một đỉnh. Mục tiêu của giải thuật là tìm ra các tập hợp các điểm ảnh nằm gần nhau và có giá trị bằng nhau. Sau đây chúng tôi sẽ trình bày hai giải thuật để đánh nhãn vùng liên thông là: một thành phần trong mỗi thời điểm (one component at a time) và hai vòng (two-pass).
\subsection{Giải thuật một thành phần trong mỗi thời điểm}
Đây là giải thuật cơ bản và dễ nghĩ tới nhất. Với giải thuật này, từng điểm ảnh sẽ được duyệt, sau đó kiểm tra tất cả các điểm liên thông và cùng giá trị với điểm ảnh này, lưu chúng lại ở dạng ma trận sao cho các thành liên thông có cùng giá trị. Đây là một phần giải thuật phân đoạn của Vincent và Soille\cite{87344}. Giải thuật với ma trận đầu vào nhị phân có hai giá trị $0, 1$ có mã giả \ref{OCAAT}.\\
\begin{algorithm}
  \caption{Giải thuật một thành phần trong mỗi thời điểm}\label{OCAAT}
  \begin{algorithmic}[1]
    \Procedure{OCAAT}{Array $A[m,n]$}
    \State Array $B[m, n]$ = $0$ \Comment{Khởi tạo ma trận trả về, 0: chưa duyệt}
    \State Queue $q$ \Comment{Khởi tạo queue q rỗng}
    \State Integer $current\_label$ = $1$
    
    \For{$i = 0$ to $m$ }
        \For{$j = 0$ to $n$}
            \If{$A[i,j]$ == $1$ and $B[i,j]$ == $0$} \Comment{điểm (i,j) không phải nền và chưa duyệt}
                \State $B[i,j]$ = $current\_label$ \Comment{đánh nhãn điểm(i,j)}
                \State $q.push([i,j])$
            \Else
                \State continue
            \EndIf
            \While {$q$ is not empty}
                \State $pixel$ = $q.pop()$
                \State $neighbors$ = $get\_neighbors(pixel)$
                \For {each $neighbor$ in $neighbors$}
                    \If{$A[neighbor]$ == $1$ and $B[neighbor]$ == $0$}
                        \State $B[i,j]$ = $current\_label$
                        \State $q.push(neighbor)$
                    \EndIf
                \EndFor
            \EndWhile
            \State $current\_label$ += $1$
        \EndFor   
    \EndFor
    \Return $B, current\_label$
    \EndProcedure
  \end{algorithmic}
\end{algorithm}
Giải thuật \ref{OCAAT} có thể mở rộng trên nhiều chiều và nhiều giá trị điểm ảnh khác nhau. Trong bài toán xác định vùng gan, chúng tôi sử dụng giải thuật đánh nhãn trong không gian 3-chiều và hai giá trị đã được làm tròn thành $0,1$.

\subsection{Giải thuật hai vòng}
Một giải thuật khác để đánh nhãn cho các vùng được trình bày trong \ref{twopass}. Tuỳ giải thuật \ref{twopass} mà hàm $find$ và $union$ sẽ được sử dụng với một cấu trúc dữ liệu đặc biệt để việc chia và gộp các vùng nhanh hơn.
\begin{algorithm}
  \caption{Giải thuật hai vòng}\label{twopass}
  \begin{algorithmic}[1]
    \Procedure{TwoPass}{Array $A[m,n]$}
    \State Array $B[m, n] = 0$ \Comment{Khởi tạo ma trận trả về, 0: chưa duyệt}
    \State Dict $dict = \{\}$
    \State Integer $next\_label = 1$
    
    \For{$i = 0$ to $m$ }
        \For{$j = 0$ to $n$}
            \If{$A[i,j] != 0$}
                \State $neighbors$ = $get\_neighbors\_foreground(A[i,j])$
                \If{$neighbors$ is empty}
                    \State $dict[next\_label]$ = $set()$
                    \State $B[i,j]$ = $next\_label$
                    \State $next\_label$ += $1$
                \Else
                    \State $L$ = $getLabels(neigbors)$
                    \State $B[i,j]$ = $min(L)$
                    \For {each $label$ in $L$}
                        \State $dict[label]$ += $union(dict[label], L)$
                    \EndFor
                \EndIf
            \EndIf
        \EndFor   
    \EndFor
    \For{$i = 0$ to $m$ }
        \For{$j = 0$ to $n$}
            \If{$A[i,j]$ != $0$}
                \State $B[i,j]$ = $find(labels[i,j]$
            \EndIf
        \EndFor   
    \EndFor
    \State \Return $B, next\_label$
    \EndProcedure
  \end{algorithmic}
\end{algorithm}
\section{Loại bỏ khối}
Với bài toán phân đoạn lá gan, để kết quả có thể sử dụng các thao tác hậu xử lý thì kết quả đó phải đạt tới một mức độ chính xác nhất định. Khi đó trên dữ liệu 3-chiều sẽ có một thành phần liên thông lớn nhất chính là lá gan, còn những thành phần liên thông khác thông thường sẽ không trùng với lá gan vì vậy sẽ bị loại bỏ. Qua thực nghiệm, thao tác loại bỏ các thành phần liên thông khác lớn nhất cho kết quả có độ chính xác cao hơn.\\
\begin{algorithm}
  \caption{Giải thuật của chúng tôi loại bỏ các thành phần nhiễu}\label{remove_components}
  \begin{algorithmic}[1]
    \Procedure{remove\_components}{Array $A[m,n]$}
    \State Array $B[m,n]$ = $0$
    \State Integer $numComponents$ = $0$
    \State $B, numComponents$ = $labels(A)$ \Comment{Sử dụng hàm đánh nhãn ở mục trước}
    \State List<Integer> $counts$
    \For{$i = 0$ to $numComponents -1$}
        \State $counts[i]$ = $count_pixel_equal(B, i)$
    \EndFor
    \State Integer $choose$ = $max(counts)$
    \For{each $P$ in $B$}
        \If{$P$ != $choose$}
            \State $P$ = $0$ \Comment{Gán điểm P là nền}
        \Else
            \State $P$ = $1$
        \EndIf
    \EndFor
    
    \State \Return $B$
    \EndProcedure
  \end{algorithmic}
\end{algorithm}



\begin{figure*}[h]
  \centering
  \subfigure[Một lát cắt dự đoán trước khi hậu xử lý]{%
    \includegraphics[width=0.45\textwidth]{Images/no_fill_result.png}%
    \label{fig:no_post_pro}%
    }
    \subfigure[Một lát cắt dự đoán sau khi hậu xử lý]{%
    \includegraphics[width=0.45\textwidth]{Images/filled_result.png}%
    \label{fig:post_pro}%
    }
  \caption{Thay đổi của một dự đoán trước và sau hậu xử lý}
  \label{fig:ab}
\end{figure*}

\section{Lấp đầy khối}
Sau khi có được kết quả dự đoán, bên cạnh loại bỏ các khối dự đoán nhiễu, chúng tôi thực hiện thao tác lấp đầy khối dự đoán chính là lá gan. Qua thực nghiệm, kết quả dự đoán sau khi được xử lý bằng cách này cho độ chính xác cao hơn và hình ảnh hiển thị rõ ràng hơn. Hiện tại đã có nhiều thư viện của python hỗ trợ thao tác này ví dụ như \textbf{scipy} với hàm \textbf{ndimage.binary\_fill\_holes}. Tuy nhiên, chúng tôi vẫn có một được một giải thuật xử lý thao tác này bằng cách sửa đổi giải thuật \ref{OCAAT}.

\begin{algorithm}
  \caption{Giải thuật của chúng tôi để lấp đầy khối}\label{OCAAT2}
  \begin{algorithmic}[1]
    \Procedure{fill\_hole}{Array $A[m,n]$}
    \State Array $B[m, n]$ = $1$ \Comment{Khởi tạo ma trận trả về, 1: chưa duyệt}
    \State Queue $q$ \Comment{Khởi tạo queue q rỗng}
    
    \For{$(i,j)$ is the address of pixels on the border} \Comment{Lấy những điểm trên biên}
        \If{$A[i,j]$ == $0$ and $B[i,j]$ == $1$} \Comment{điểm (i,j) là nền và chưa duyệt}
            \State $B[i,j]$ = $0$ \Comment{ghi nhận điểm nền}
            \State $q.push([i,j])$
        \Else
            \State continue
        \EndIf
        \While {$q$ is not empty}
            \State $pixel$ = $q.pop()$
            \State $neighbors$ = $get\_neighbors(pixel)$
            \For {each $neighbor$ in $neighbors$}
                \If{$A[neighbor]$ == $0$ and $B[neighbor]$ == $1$}
                    \State $B[i,j]$ = $0$
                    \State $q.push(neighbor)$
                \EndIf
            \EndFor
        \EndWhile
    \EndFor
    \State \Return $B, current\_label$
    \EndProcedure
  \end{algorithmic}
\end{algorithm}
\chapter{Thử nghiệm mô hình}
Chúng tôi đã tiến hành xây dựng và thử nghiệm phương pháp phân đoạn lá gan trên các mô hình học sâu khác nhau. Mỗi mô hình đều được chúng tôi kiểm tra trên ít nhất một hệ thống chấm điểm công khai trung lập.
\section{Mô hình đảo tích chập 2-chiều}
\subsection{Chuẩn bị dữ liệu huấn luyện}
Khi huấn luyện mô hình này chúng tôi thực hiện phương án làm giàu dữ liệu trực tuyến, tức là các hình ảnh sẽ được làm giàu trong lúc huấn luyện. Với sự hỗ trợ của các nền tảng học máy, phương án làm giàu dữ liệu này sẽ tiết kiệm được không gian đĩa giành cho lưu dữ liệu. Bên cạnh đó, vì việc huấn luyện mạng sử dụng GPU, nên các nền tảng đã thiết kế để có thể chuẩn bị dữ liệu bằng CPU và RAM tránh tình trạng dùng chung tài nguyên gây tranh chấp và thời gian huấn luyện bị kéo dài.\\
Chúng tôi thực hiện các phép làm giàu dữ liệu cơ bản gồm:
\begin{itemize}
    \item Thực hiện phép quay từ -5 đến 5 độ quan tâm ảnh.
    \item Thực hiện phép phóng to từ 0.5 (làm nhỏ đi hai lần) đến 1.5. Phép phóng to này không bắt buộc hai chiều toạ độ cùng một độ phóng.
\end{itemize}
Chúng tôi huấn luyện mô hình này bằng tập dữ liệu SLiver07 gồm 20 ảnh CT 3-chiều được chia ra theo tỉ lệ 19:1 cho huấn luyện và đánh giá. Mô hình được kiểm tra trên 10 ảnh 3-chiều tại \cite{website:slvier07}.
\subsection{Xây dựng mô hình mạng}
Chúng tôi sử dụng mô hình mạng như \cite{unpoolref}. Vì mô hình này ban đầu sử dụng cho ảnh 2 chiều kích thước 224x224 với tập dữ liệu VOC2012 nên để phù hợp với bài toán, chúng tôi đã thử thay đổi các thông số của mạng khác nhau như số khối, số lớp tích chập - đảo tích chập, số lõi cũng như kích thước lõi mối lớp tích chập, ... Kết quả được mô hình tốt nhất là mô hình chỉ thay đổi kích thước ảnh đầu vào và không dùng các trọng số đã huấn luyện (pretrain).
\paragraph{Thông số huấn luyện mô hình}
\begin{itemize}
\item Máy tính sử dụng: Máy ảo Google Colab với card đồ họa GPU Tesla T4.
\item Framework: Tensorflow 1.13.1.
\item Ngôn ngữ: Python 3.6.4.
\item Hàm mất mát (Loss function): Cross entropy loss.
\item Phương pháp tối ưu: Adam optimizer.
\item Các siêu tham số: Learning rate 0.0005.
\end{itemize}
\paragraph{Kết quả}
Với mô hình này chúng tôi kiểm tra kết quả trên hệ thống chấm điểm của SLiver07 được kết quả như bảng sau:
\begin{table}[h]
\begin{center}
\begin{tabular}{|@{\hspace{1cm}}c@{\hspace{1cm}}|@{\hspace{1cm}}c@{\hspace{1cm}}|@{\hspace{1cm}}c@{\hspace{1cm}}|}
\hline
\textbf{Độ Đo}     & \textbf{Giá trị}      & \textbf{Điểm} \\ \hline
VOE            & 6.63                      & 74.11         \\ \hline
RVD            & -0.31                     & 83.81         \\ \hline
AvgD           & 1.15                      & 71.21         \\ \hline
RMSD           & 2.64                      & 65.12         \\ \hline
MaxD           & 26.49                     & 63.33         \\ \hline
Average Score  & --                        & 71.51         \\ \hline
\end{tabular}
\caption{\label{tab:deconv_SLIVER07}Kết quả đánh giá mô hình đảo tích chập trên SLIVER07}
\end{center}
\end{table}

\section{Mô hình mạng nơ-ron tích chập 3D (CNN)}
\subsection{Chuẩn bị dữ liệu huấn luyện}
Phương pháp làm giàu giữ liệu:
\begin{itemize}
\item Thực hiện phép quay ngẫu nhiên từ -10 đến 10 độ quanh trục X.
\item Thực hiện phép quay ngẫu nhiên từ -10 đến 10 độ quanh trục Y.
\item Thực hiện phép quay ngẫu nhiên từ -45 đến 45 độ quanh trục Z.
\item Thực hiện phép biến đổi elastic với $\alpha = 1024$, $\alpha_{affine} = 40$, $\sigma = 40 $.
\item Cắt khối ngẫu nhiên với kích thước 192x192x64.
\end{itemize}
Dữ liệu từ các tập SLiver07 \cite{website:slvier07}, 3Dircadb \cite{website:data_3DIRCADb}, LiTS2017 \cite{website:LiTS} gồm 171 ảnh CT được chia làm hai phần 121 ảnh phần huấn luyện và 50 ảnh CT phần đánh giá :
\begin{itemize}
\item Phần huấn luyện (Training data): Đây là phần dùng để huấn luyện mô hình. Dữ liệu huấn luyện bao gồm 10080 mẫu dữ liệu với kích thước 192x192x64.
\item Phần đánh giá (Validation data): Đây là phần dùng để kiểm tra nếu mô hình quá khớp (Overfitting). Dữ liệu đánh giá bao gồm 1080 mẫu dữ liệu với kích thước 192x192x64.
\end{itemize}
\subsection{Xây dựng mô hình mạng nơ-ron tích chập 3 chiều}
Dựa trên kiến trúc mạng nơ-ron tích chập 3 chiều và phương pháp huấn luyện giám sát sâu\cite{dsn_paper} của Qi Dou và cộng sự, chúng tôi đã xây dựng mô hình phù hợp và có thể huấn luyện trên máy ảo google colab (Hình \ref{our_CNN_org}). Mô hình này gồm hai phần: phần mã hóa và phần giải mã. Trong phần mã hóa, chúng tôi đã sử dụng bốn lớp Max pooling với stride 2x2x2 và các lớp tích chập với activation ELU. Sau phần mã hóa, kích thước bản đồ đặc trưng đã giảm kích thước đầu vào 192x192x64x1 còn 12x12x4x256. Phần giải mã chỉ sử dụng bốn lớp đảo tích chập với stride 2x2x2 và một lớp sigmoid. Đầu ra của lớp Sigmoid là một bản đồ xác xuất có kích thước 192x192x64x1. Mỗi giá trị trong bản đồ xác xuất này là xác xuất mà voxel tương ứng ở đầu vào là gan.\\


\begin{figure}[h]
\centering
    \includegraphics[totalheight=15cm]{Images/CNN_org.jpg}
    \caption{Kiến trúc mạng nơ-ron tích chập 3-chiều đề xuất}
    \label{our_CNN_org}
\end{figure}
\subsection{Huấn luyện mạng và kết quả thực nghiệm}
\paragraph{Phương pháp huấn luyện giám sát sâu:}
Mạng nơ-ron tích chập 3D được biến đổi và huấn luyện theo phương pháp giám sát sâu. Theo phương pháp này, chúng tôi đã thêm một số phần giải mã (decoder) thích hợp vào mạng nơ-ron như Hình \ref{our_CNN}. Hàm mất mát (loss function) sẽ được tính từ hàm mất mát của tất cả các phần giải mã:\\
\begin{center} $Loss = \mu_1 Loss_{aux1} + \mu_2 Loss_{aux2}+\mu_3 Loss_{aux3}+\mu_4 Loss_{main}$\end{center}
\begin{figure}[h]
\centering
    \includegraphics[totalheight=15cm]{Images/CNN.jpg}
    \caption{Kiến trúc mạng nơ-ron tích chập 3-chiều giám sát sâu đề xuất. Ba phần giải mã phụ được thêm vào trước lớp Max pooling.}
    \label{our_CNN}
\end{figure}
\paragraph{Thông số huấn luyện mô hình}
\begin{itemize}
\item Máy tính sử dụng: Máy ảo Google Colab với card đồ họa GPU Tesla T4.
\item Framework: Tensorflow 1.13.1.
\item Ngôn ngữ: Python 3.6.4.
\item Hàm mất mát (Loss function): Cross entropy loss.
\item Phương pháp tối ưu: Adam optimizer.
\item Các siêu tham số: Learning rate 0.0001, L2 regularization 0.0001.
\item Các tham số tổng hợp hàm mất mát: $\mu_1 = 1$,  $\mu_2 = 2$,  $\mu_3 = 4$,  $\mu_4 = 8$
\end{itemize}
\paragraph{Kết quả phân đoạn}
Mô hình CNN được đánh giá trên các hệ thống đánh giá kết quả tự động mang tính khách quan cao trên tập kiểm thử SLIVER07 và LITS2017 (Bảng \ref{tab:CNN-SLIVER07_Test}). Ngoài ra chúng tôi đã tự đánh giá mô hình này trên tập dữ liệu 3Dircadb\_1. Phương pháp này đã đạt kết quả khá tốt trên tập SLIVER07 và 3Dircadb với độ đo VOE thấp hơn 6\%. Đây là một kết quả có thể chấp nhận được và sử dụng cho việc xây dựng mô hình lá gan. Tuy nhiên, khả năng hân đoạn của mô hình CNN còn khá hạn chế trên tập kiểm thử LiTS2017.
\begin{table}[]
\begin{tabular}{|c|c|c|c|c|}
\hline
\textbf{Độ Đo} & \textbf{SLIVER07 (train)} & \textbf{SLIVER07 (test)} & \textbf{3Dircadb\_b1} & \textbf{LiTS2017 (test)} \\ \hline
VOE            & 3.72                      & 5.58                     & 4.69                  & 8.80                     \\ \hline
RVD            & 0.13                      & -0.19                    & -0.26                 & 2.60                     \\ \hline
AvgD           & 0.49                      & 0.86                     & 0.59                  & 1.44                     \\ \hline
RMSD           & 1.03                      & 1.77                     & 1.21                  & 2.82                     \\ \hline
MaxD           & 14.00                     & 20.79                    & 17.41                 & 24.94                    \\ \hline
Score          & 87.54                     & 79.89                    & 83.98                 & \_                       \\ \hline
Dice per case  & 98.10                     & \_                       & 97.60                 & 95.3                     \\ \hline
Dice global    & \_                        & \_                       & \_                    & 95.8                     \\ \hline
\end{tabular}
\caption{\label{tab:CNN-SLIVER07_Test}Kết quả đánh giá mô hình CNN trên ba tập dữ liệu SLIVER07, 3Dircadb và LiTS2017.}
\end{table}

\section{Mô hình Unet 3D}
\subsection{Phương pháp Tranfer Learning cho bài toán phân đoạn ảnh}
Tranfer learning hay Fine-tuning là phương pháp sử dụng lại trọng số của một số lớp, hoặc toàn bộ các lớp của một mô hình đã được huấn luyện trước đó để khởi tạo cho các lớp trên mô hình mới. Mô hình mới sẽ được huấn luyên với dữ liệu khác có tính tương quan cao hoặc khác hoàn toàn với nguồn dữ liệu cũ.
Phương pháp huấn luyện với mô hình mới phụ thuộc vào độ tương quan giữa dữ liệu cũ và dữ liệu mới và kích thước của dữ liệu mới:
\begin{itemize}
\item Dữ liệu mới là nhỏ hơn và tương tự như dữ liệu gốc: vì dữ liệu mới nhỏ và có độ tương quan cao với dữ liệu cũ nên việc huấn luyện lại mô hình dễ dẫn đến hiện tượng quá khớp (overfitting). Vì vậy, chỉ huấn luyện lại phần giãi mã dựa trên đặc trưng thu được qua phần mã hóa gốc.
\item Dữ liệu mới là lớn và tương tự như dữ liệu gốc: vì dữ liệu mới lớn, hiện tượng quá khớp (overfitting) khó có khả năng xảy ra nên có thể huấn luyện toàn bộ mô hình với learning rate nhỏ.
\item Dữ liệu mới là nhỏ và rất khác so với dữ liệu gốc: sử dụng bộ giải mã đơn giản và chỉ huấn luyện một số lớp cuối cùng.
\item Dữ liệu mới là lớn và rất khác so với dữ liệu gốc: có thể khởi tạo từ bộ trọng số cũ, không cần phải huấn luyện lại từ đầu.
\paragraph{Dữ liệu huấn luyện Unet 3-chiều} Chúng tôi tạo ra ba bộ dữ liệu riêng biệt từ mỗi tập SLIVER07, 3Dircadb và LiTS2017 với phương pháp làm giàu như dữ liệu huấn luyện mô hình CNN. Mỗi bộ dữ liệu mới này sẽ nhỏ hơn rất nhiều so với dữ liệu huấn luyện mô hình CNN (~2000 mẫu/1 bộ so với 10080 mẫu dữ liệu cũ).
\paragraph{Sử dụng phương pháp Tranfer Learning trên mô hình Unet 3-chiều} Chúng tôi đã sử dụng lại toàn bộ phần mã của mô hình CNN cho mô hình Unet 3-chiều. Chúng tôi chỉ thực hiện huấn luyện phần giải mã và nối tắt với mỗi bộ dữ liệu.
\end{itemize}
\subsection{Xây dựng mô hình mạng Unet 3-chiều}
Mô hình Unet 3-chiều này được xây dựng dựa trên cơ sở mô hình Unet 3-chiều \cite{3DUnet} của Ozgun Cikek và cộng sự và một số thay đổi ở cả ba phần giải mã, mã hóa và nối tắt. (Hình \ref{Unet3D_pro})
\paragraph{Phần mã hóa} Phần mã hóa của mô hình Unet 3-chiều này sử dụng lại kiến trúc và trọng số phần mã hóa của mô hình tích chập 3 chiều (CNN) mà chúng tôi đã đề cập ở phần trước.
\paragraph{Phần nối tắt} Chúng tôi đã thêm một lớp tích chập với kích thước cửa sổ 5x5 vào phần nối tắt để giảm số kênh (channel) trước khi kết hợp với kết quả lớp đảo tích chập tương ứng.
\paragraph{Phần giải mã} Phần giải mã được chỉnh sửa số cửa sổ trên các lớp tích chập, đảo tích chập dể phù hợp với phần mã hóa.
\begin{figure}[h]
\centering
    \includegraphics[totalheight=23cm]{Images/UNET3D_pro.jpg}
    \caption{Kiến trúc mô hình Unet 3-chiều đề xuất}
    \label{Unet3D_pro}
\end{figure}
\subsection{Huấn luyện mạng và kết quả}
\paragraph{Thông số huấn luyện mô hình}
\begin{itemize}
\item Máy tính sử dụng: Máy ảo Google Colab với card đồ họa GPU Tesla T4.
\item Framework: Tensorflow 1.13.1.
\item Ngôn ngữ: Python 3.6.4.
\item Hàm mất mát (Loss function): Cross entropy loss.
\item Phương pháp tối ưu: Adam optimizer.
\item Các siêu tham số: Learning rate 0.0001.
\end{itemize}
\paragraph{Kết quả phân đoạn}
Chúng tôi thực hiện đánh giá mô hình Unet 3-chiều trên tập dữ liệu tương ứng đã sử dụng khi huấn luyện (Ví dụ: Nếu huấn luyện mô hình với dữ liệu được làm giàu từ tập LITS2017 thì sẽ đánh giá trên tập kiểm thử của tập này). Mô hình Unet 3-chiều đã đạt kết quả rất tốt trên cả ba tập SLIVER07 (tập huấn luyện), 3Dircadb\_1 và LiTS2017 (tập kiểm thử). 
\begin{table}[]
\begin{tabular}{|c|c|c|c|}
\hline
\textbf{Độ Đo} & \textbf{SLIVER07 (train)} & \textbf{3Dircadb\_b1} & \textbf{LiTS2017 (test)} \\ \hline
VOE            & 2.67                      & 2.57                  & 7.60                     \\ \hline
RVD            & 0.39                      & -0.14                 & 0.02                     \\ \hline
AvgD           & 0.32                      & 0.26                  & 1.20                     \\ \hline
RMSD           & 0.65                      & 0.55                  & 2.54                     \\ \hline
MaxD           & 8.26                      & 7.82                  & 23.92                    \\ \hline
Score          & 91.81                     & 92.61                 & \_                       \\ \hline
Dice per case  & 98.65                     & 98.69                 & 96.00                    \\ \hline
Dice global    & \_                        & \_                    & 96.60                     \\ \hline
\end{tabular}
\caption{\label{tab:CNN-SLIVER07_Test}Kết quả đánh giá mô hình UNET3D trên ba tập dữ liệu SLIVER07, 3Dircadb và LiTS2017.}
\end{table}




\chapter{Đánh gía kết quả}
\chapter{Xây dựng ứng dụng}
Để hiển thị kết quả của mô hình một cách trực quan thì nhóm đã xây dựng một ứng dụng website cho phép người dùng upload dữ liệu, sau đó hiển thị kết quả phân đoạn và mô hình 3D. Phần này sẽ trình bày về cấu trúc và các chức năng chính của ứng dụng.
\section{Cấu trúc ứng dụng}
Ứng dụng được xây dựng theo mô hình client - server API(gồm hai phần là client và server riêng biệt hoạt động với nhau qua cơ chế gọi API). 
Mô hình chung của ứng dụng được mô tả như sau:
\begin{figure}[h]
\centering
    \includegraphics[totalheight=7cm]{Images/app_struct.png}
    \caption{Các thành phần của ứng dụng}
    \label{skip_conn}
\end{figure}
\subsection{RESTful API}
REST(Representational State Transfer) ra đời vào năm 2000 bởi Roy Thomas Fielding có thể gọi đây là các ràng buộc và quy ước mà khi hệ thống nào làm theo thì được gọi là REST.\\
RESTful API là một tiêu chuẩn dùng trong việc thết kế các thiết kế API cho các ứng dụng web để quản lý các resource. RESTful là một trong những kiểu thiết kế API được sử dụng phổ biến nhất ngày nay.\\
Cấu trúc của một RESTful API gồm những phần như sau:
\begin{itemize}
    \item METHOD(Bắt buộc): Gồm các phương thức cơ bản như GET, POST, PUT, DELETE.
    \item URL(Bắt buộc): đường dẫn của API(đường dẫn này sẽ được phân chia trong Routers bên phía server).
    \item DATA(tùy chọn): Đối tượng JSON mô tả dữ liệu kèm theo gửi lên server.
    \item HEADER(Tùy chọn): Một đối tượng JSON thường dùng trong việc bảo mật và xác thực người dùng...
    \item PARAM: Các tham số đi kèm sau dấu "?" của URL.
\end{itemize}
Các bước gọi và phản hồi API cơ bản như sau:\\
Phía client tạo một API chứa những thông tin cần thiết như trên, phía server sẽ kiểm tra header có phù hợp hay không, sau đó nhận dữ liệu, kiểm tra tính hợp lệ của dữ liệu, xử lý và trả lại dữ liệu cần thiết cho client nếu có.
\begin{figure}[h]
\centering
    \includegraphics[totalheight=7cm]{Images/app_json.png}
    \caption{Mô tả đơn giản luồng xử lý của một API}
    \label{skip_conn}
\end{figure}
\subsection{Server}
Phần server chính của ứng dụng(Django server) được xây dựng bằng ngôn ngữ Python với framework là Django, làm việc trực tiếp với mô hình đã huấn luyện được và server này sẽ cung cấp một API để client gọi vào. Về framework Django thì chỉ sử dụng để cung cấp và phục vụ một API nên trong báo cáo sẽ không trình bày về các đặc điểm của framework này. Luồng xử lý của server được mô tả như hình bên dưới.
\begin{figure}[h]
\centering
    \includegraphics[totalheight=7cm]{Images/app_django_struct.png}
    \caption{Luồng xử lý trong Django server}
    \label{skip_conn}
\end{figure}
\paragraph{Quá trình xử lý dữ liệu đầu vào và chạy model\\}

Dữ liệu đầu vào để server xử lý phải là dữ liệu dưới định dạng raw/mhd(định dạng như tập Sliver07). Dữ liệu có thể chỉ cần tập ảnh scan nếu muốn phân đoạn hoặc bao gồm cả nhãn nếu muốn so sánh và đánh giá kết quả phân đoạn.
Sau khi nhận dữ liệu đầu vào sẽ tiến hành chạy model để cho ra kết quả phân đoạn(Kết quả 1). Đầu ra của quá trình này cũng là file dưới định dạng raw/mhd. Một vài thông tin về quá trình chạy model như sau(cấu hình máy local kèm theo):
\begin{figure}[h]
\centering
    \includegraphics[totalheight=7cm]{Images/app_localinfor.png}
    \caption{Thời gian chạy model của máy tương ứng}
    \label{skip_conn}
\end{figure}
\paragraph{Quá trình hậu xử lý\\}
Tất các các hình ảnh đầu ra đều có định dạng là png.
\begin{itemize}
    \item Với tập đầu vào chỉ có scan, kết quả của quá trình này sẽ bao gồm: Tập ảnh scan, tập ảnh kết quả phân đoạn, tập ảnh phân đoạn dán lên scan, file csv gồm thông tin tập đỉnh và mặt của mô hình gan 3D, thể tích gan theo kết quả phân đoạn.
    \item Với tập có cả scan và nhãn thì kết quả bao gồm: Tập ảnh scan, tập ảnh nhãn, tập ảnh kết quả phân đoạn so sánh với nhãn,  tập ảnh kết quả phân đoạn so sánh với nhãn dán lên scan, file csv gồm thông tin tập đỉnh và mặt mô hình gan 3D và thể tích gan của kết quả phân đoạn và của nhãn.
\end{itemize}
Sau khi tạo các thông tin đó thì tạo file zip kèm theo cho mỗi tập.
\begin{figure}[h]
\centering
    \includegraphics[totalheight=7cm]{Images/app_scancompare.png}
    \caption{Ảnh phân đoạn khi có nhãn}
    \label{skip_conn}
\end{figure}
\begin{figure}[h]
\centering
    \includegraphics[totalheight=7cm]{Images/app_scanoverlap.png}
    \caption{Ảnh phân đoạn khi có nhãn dán lên scan}
    \label{skip_conn}
\end{figure}
\paragraph{Tạo đối tượng JSON\\}
Với cơ chế gọi API, dữ liệu nhận và gửi sẽ ở định dạng json nên dữ liệu trả về cho client sẽ là một đối tượng json với các trường dữ liệu được mô tả như sau (các đường dẫn tương ứng là các file đã được lưu trong server):
\begin{itemize}
 \item ListPredict: mảng các đường dẫn của ảnh kết quả phân đoạn.
 \item ListLabel: mảng các đường dẫn của ảnh nhãn(nếu có).
 \item ListOverlap: mảng các đường dẫn ảnh kết quả dán lên scan.
 \item CsvFileLabel: đường dẫn tới file csv của nhãn(nếu có).
 \item CsvFilePredict: đường dẫn tới file csv kết quả phân đoạn.
 \item LinkZip: đối tượng JSON chứa các đường dẫn tới file zip của các tập.
 \item VolumePredict: thông tin thể tích theo kết quả phân đoạn.
 \item VolumeLabel: thông tin thể tích nhãn(nếu có).
 \item 
\end{itemize}
Sau khi tạo xong đối tượng JSON sẽ được gửi về client xử lý.
\subsection{Client}
Với cấu trúc server như trên chúng ta có thể xây dựng phần client một cách linh động sao cho trực quan nhất trên bất cứ nền tảng nào qua việc gọi API. Ở phần này thì nhóm xây dựng phần client trên nền tảng website sử dụng framework Beego, với ngôn ngữ phía server là Golang, phía client sử dụng framework là Angularjs để giao tiếp với Golang server.
\begin{figure}[h]
\centering
    \includegraphics[totalheight=7cm]{Images/app_beego_struct.png}
    \caption{Cấu trúc phần client chính}
    \label{skip_conn}
\end{figure}
\paragraph{Giới thiệu về Golang\\}
Golang hay còn được gọi là Go được thiết kế và phát triển bởi Robert Griesemer, Rob Pike và Ken Thompson tại google vào năm 2007, được giới thiệu vào năm 2009 và hiện tại được sử dụng trong hầu hết trong các sản phẩm của Google. Hiện tại thì Go khá được ưa chuộng bởi các đặc điểm sau:
\begin{itemize}
    \item Hỗ trợ khai báo kiểu dữ liệu động.
    \item Tốc độ biên dịch nhanh.
    \item Hỗ trợ các tác vụ đồng thời.
    \item Ngôn ngữ đơn giản, ngắn gọn.
\end{itemize}
Nói tóm lại, Go có tốc độ xử lý nhanh tương đương như C/C++ nhưng cấu trúc đơn giản và hỗ trợ trình dọn rác tự động nên dễ sử dụng, hiệu quả cao. Để biết thêm thông tin về Go thì có thể tham khảo tại đường dẫn.

\paragraph{Giới thiệu về framework Beego\\}
Beego là một framework open source cho cộng đồng sử dụng miễn phí và phát triển nó theo đường dẫn https://github.com/astaxie/beego.\\
Ta sử dụng beego để tạo một ứng dụng web chạy trực tiếp trên máy chủ local. Khi tạo một dự án bằng Beego ta sẽ được một thư mục có cấu trúc rõ ràng.\\
(Ảnh cây thư mục trong beego).\\
Ứng dụng web được tạo bởi Beego theo mô hình MVC(Model-View-Controller) và giao tiếp qua cơ chế gọi API, các thành phần chính của ứng dụng được giải thích ngắn gọn như sau:\\
\begin{itemize}
    \item Conf: chứa file config định nghĩa một số thông tin chính như cổng, tên, thông tin Database…
    \item Controllers: Nhận và xử lý dữ liệu từ bên ngoài, rồi gọi tới Model và gửi dữ liệu ra nếu có.
    \item Main.go: File để chạy ứng dụng
    \item Models: Nhận dữ liệu đã hợp lệ từ Controllers để xử lý, thường là các tác vụ liên quan tới Database và trả về dữ liệu cần thiết cho Controller.
    \item Routers: Quản lý, phân chia các API cho các Controller tương ứng.
    \item Static: chứa các dữ liệu cần thiết cho phía client.
    \item Tests: Dùng để kiểm tra lại ứng dụng, nơi làm việc của tester.
    \item Views: Mặc định chứa file hiển thị trang chủ.
\end{itemize}
Để tìm hiểu kĩ hơn về framework Beego có thể tham khảo tại: https://beego.me/docs
\paragraph{ Giới thiệu về AngularJS\\}
Angularjs có thể gọi là một framework được viết bằng Javascript. AngularJS đưa ra hướng dẫn cụ thể trên mã lệnh HTML với các tiền tố "ng-" hay còn được gọi là directives.\\
Để sử dụng thì chỉ cần nhúng đường dẫn tới file angular.min.js như sử dụng các thư viện khác.
Một số directive quan trọng có thể kể đến như sau:
\begin{itemize}
    \item ng-app: Đánh dấu thẻ HTML mà AngularJS được bắt đầu sử dụng.
    \item ng-model: giá trị HTML trong thẻ này tương đương với biến \$scope trong phần script.
    \item ng-change, ng-click: bắt sự kiện thay đổi, click để gọi hàm \$scope tương ứng.
\end{itemize}
Một điều quan trọng nữa là AngularJS hỗ trợ gọi API để gửi và nhận dữ liệu một các dễ dàng qua biến \$http.\\
Để biết thêm về AngularJS có thể tham khảo tại 

\paragraph{Một số thư viện hỗ trợ khác được sử dụng\\}
\begin{itemize}
    \item Bootstrap: Thư viện quen thuộc để xây dựng giao diện responsive dễ dàng. Trong ứng dụng này sử dụng bootstrap phiên bản 3.3.7.
    \item Plotly: Thư viện giúp hiển thị hình ảnh 3D từ file csv chứa tập đỉnh và mặt.
    \item Jquery, Font Awesome, Rangesslider hỗ trợ việc hiển thị được tốt và hiệu quả hơn.
\end{itemize}

\paragraph{Luồng xử lý bên phía client của ứng dụng\\}
Như đã nên trên, bên phía client của ứng dụng cũng sẽ có client-server nữa để thực hiện việc hiển thị và lưu lại kết quả gần nhất. Ta tạm gọi server chính là Django-server và server phụ là Beego-server. Luồng xử lý của phần client này được mô tả như sau:\\
Sau khi upload dữ liệu cho Django-server qua một API, client sẽ nhận về các thông tin của server này đồng thời gửi lại cho Beego-server, sau đó Beego-server sẽ lấy những đường dẫn chứa file zip giải nén và lưu vào thư mục của nó đồng thời tạo file lưu thông tin thể tích lá gan sau đó gửi cho client đường dẫn là các file được lưu trong Beego-server. Client sẽ thực hiện việc hiển thị kết quả từ những đường dẫn này. Khi khởi động ứng dụng lên, client sẽ gọi API cho Beego-server để lấy lại những đường dẫn này. Sau khi có được đường dẫn thì hiển thị kết quả, dùng thư viện đọc file để hiển thị mô hình 3D. Phần này sẽ được trình bày trực quan ở mục 9.2.
\section{Các chức năng và kết quả hiển thị của ứng dụng.}
Về phần hiển thị, trang web được xây dựng khá đơn giản nhưng đảm bảo được những thông tin cần thiết của kết quả phân đoạn một lá gan.
\subsection{Trang giới thiệu.}
Trang này chỉ đơn giản là giới thiệu về chức năng của website.
\begin{figure}[h]
\centering
    \includegraphics[totalheight=7cm]{Images/app_intro.png}
    \caption{Giao diện trang giới thiệu ứng dụng}
    \label{skip_conn}
\end{figure}
\subsection{Trang upload dữ liệu}
Trang này cho phép người dùng upload dữ liệu lên. Dữ liệu là tập scan có định dạng raw/mhd được nén trực tiếp ở dạng zip. Sau khi upload thì đợi kết quả trả về(chạy trên máy local mất khoảng 30 phút). Nếu dữ liệu có thêm label thì kết quả trả về có dạng để so sánh. Kết quả sẽ được hiển thị ở trang tiếp theo.
\begin{figure}[h]
\centering
    \includegraphics[totalheight=7cm]{Images/app_upload.png}
    \caption{Giao diện trang upload dữ liệu}
    \label{skip_conn}
\end{figure}
\subsection{Trang hiển thị kết quả phân đoạn}
Như đã nêu trên, nếu dữ liệu upload chỉ có scan, thì trang này hiển thị 3 mục như sau:
\begin{itemize}
    \item Ảnh scan(ảnh gốc lúc upload).
    \item Ảnh phân đoạn gan.
    \item Ảnh phân đoạn dán lên scan tương ứng.
\end{itemize}

\begin{figure}[h]
\centering
    \includegraphics[totalheight=7cm]{Images/app_scan.png}
    \caption{Hiển thị ảnh scan}
    \label{skip_conn}
\end{figure}
\begin{figure}[h]
\centering
    \includegraphics[totalheight=7cm]{Images/app_label.png}
    \caption{Hiển thị ảnh phân đoạn khi không có nhãn}
    \label{skip_conn}
\end{figure}
\begin{figure}[h]
\centering
    \includegraphics[totalheight=7cm]{Images/app_overlap_1.png}
    \caption{Hiển thị ảnh phân đoạn dán lên scan}
    \label{skip_conn}
\end{figure}
Nếu tập dữ liệu có thêm nhãn thì kết quả hiển thị có ảnh scan giống với mục trên, các ảnh còn lại sẽ là kết quả để so sánh với nhãn.
\begin{figure}[h]
\centering
    \includegraphics[totalheight=7cm]{Images/app_showscanompare.png}
    \caption{Hiển thị ảnh phân đoạn so sánh với nhãn}
    \label{skip_conn}
\end{figure}
\begin{figure}[h]
\centering
    \includegraphics[totalheight=7cm]{Images/app_labelreal.png}
    \caption{Hiển thị ảnh nhãn}
    \label{skip_conn}
\end{figure}
\begin{figure}[h]
\centering
    \includegraphics[totalheight=7cm]{Images/app_overlap_2.png}
    \caption{Hiển thị ảnh phân đoạn so sánh với nhãn dán lên scan}
    \label{skip_conn}
\end{figure}

\subsection{Trang hiển thị mô hình 3D}
Trang này hiển thị mô hình 3D bằng cách dùng thư viện plotly.js đọc file csv được tạo ra trong Django server đồng thời hiển thị thể tích của gan. Người dùng có thể dùng chuột để xoay, thu nhỏ, phóng to để xem chi tiết mô hình.
\begin{figure}[h]
\centering
    \includegraphics[totalheight=7cm]{Images/app_3dpredict.png}
    \caption{Hiển thị mô hình 3D dự đoán kèm thể tích}
    \label{skip_conn}
\end{figure}
\begin{figure}[h]
\centering
    \includegraphics[totalheight=7cm]{Images/app_3dlabel.png}
    \caption{Hiển thị mô hình 3D của nhãn kèm thể tích}
    \label{skip_conn}
\end{figure}

Ta có thể click vào nút "Auto rotate" để lá gan xoay tự động. Một mô hình sẽ có khoảng hai trăm nghìn điểm và bốn trăm nghìn mặt nên chức năng tự động xoay khá chậm.
\section{Ưu nhược điểm của ứng dụng}
Ưu điểm:\\
\begin{itemize}
    \item Các thành phần được xây dựng riêng biệt với nhau nên dễ dàng sửa đổi và phát triển.
    \item Mô hình 3D dễ dàng điều chỉnh để xem chi tiết lá gan.
\end{itemize}
Nhược điểm:\\
\begin{itemize}
    \item Chưa có bảo mật.
    \item Giao diện còn khá đơn giản.
\end{itemize}

\chapter{Tổng kết}
% \section{Luồng xử lý dữ liệu hoàn chỉnh}
\section{Kết quả đạt được}
Với nỗ lực của các thành viên trong nhóm, đề tài đã đạt được những kết quả khả quan:
\begin{itemize}
    \item Phân đoạn gan: Đề tài đã xây dựng thành công một mô hình phân đoạn gan với độ chính xác rất cao, sẽ là công cụ hỗ trợ rất tốt cho việc xem ảnh chụp CT và chuẩn đoán các bệnh trên gan của các bác sĩ, đồng thời làm tiền đề để xây dựng mô hình phát hiện các bất thường trong lá gan từ ảnh CT.
    \item Các giải thuật tiền, hậu xử lý dữ liệu: Đề tài cũng cung cấp được các giải thuật về tiền và hậu xử lý dữ liệu khi huấn luyện. Tiền xử lý giúp việc huấn luyện thuận lợi và hậu xử lý giúp cho kết quả hiển thị được mượt mà hơn. Các giải thuật này cũng góp phần đáng kể trong việc tăng độ chính xác cho toàn mô hình. Đồng thời các đề tài khác cũng có thể áp dụng với một vài chỉnh sửa nhỏ tùy vào tập dữ liệu.
    \item Làm giàu dữ liệu: Đề tài cung cấp các giải thuật làm giàu bộ dữ liệu ảnh y khoa, làm cho mô hình dự đoán chính xác hơn, tránh overfitting. Các tập dữ liệu có nhãn hiếm hoi của những đề tài khác có thể áp dụng được những giải thuật này hoặc dựa vào đó để phát triển thêm các phương án làm giàu khác tùy thích, từ đó một phần giải quyết vấn đề khan hiếm dữ liệu có nhãn trong lĩnh vực học sâu.
    \item Đánh giá trên nhiều tập dữ liệu: Đề tài này sử dụng cả 3 tập dữ liệu có nhãn phổ biến nhất về lá gan từ trước đến nay để đánh giá độ chính xác của mô hình mà nhóm xây dựng đó là 3Dircadb, SLIVER07 và LITS2017. Trong đó tập 3Dircadb đạt độ chính xác gần 99\% với độ đo ``Dice per case``, tập SLIVER07 xếp hạng tối thiểu là thứ 16 (kết quả này cho mô hình trước đó, sau này không cho submit vì không có paper công khai) với điểm số trung bình là 79.89 trên ``SLIVER07 Grand Challenge 2007`` và tập LITS2017 đạt top 5 tại ``Liver Tumor Segmentation Challenge 2017``. Đây đều là những kết quả đạt được khá ấn tượng.
    \item Xây dựng ứng dụng hiển thị: Cuối cùng, đề trực quan hóa kết quả đạt được, nhóm đã xây dựng được một ứng dụng cho phép đăng tải dữ liệu ảnh CT và nhận kết quả phân đoạn đồng thời hiển thị mô hình 3D và tính thể tích gan tương ứng cho dữ liệu đó. Tất cả các kết quả đều được hiển thị trên giao diện web - hiện nay là phương án tiếp cận phổ biến, dễ sử dụng, dễ tuỳ chỉnh và hiệu quả nhất cho các ứng dụng nhỏ.
\end{itemize}
\section{Hạn chế và hướng phát triển}
Tuy kết quả của đề tài này thực sự rất tốt với những yêu cầu đã đề ra ban đầu, nhưng vẫn còn một số hạn chế như sau:
\begin{itemize}
    \item Đề tài này cần kết hợp với những đề tài khác như phân đoạn mạch máu và dự đoán bất thường trong gan, khi đó mới đem lại ý nghĩa thực tiễn to lớn trong việc xem và chuẩn đoán các bệnh về gan của các bác sĩ, đồng thời giúp các bác sĩ ra quyết định trong việc cắt và ghép lá gan qua ảnh CT (điều này nhóm nhận ra qua buổi hướng dẫn của các bác sĩ tại Bệnh viện Đại học Y Dược).
    \item Ứng dụng xây dựng chỉ áp dụng cho tập dữ liệu có định dạng ảnh CT chuẩn có thể đọc được bằng thư viện SimpleItk như mhd|raw, nii, niiz.
\end{itemize}
Với những nhược điểm và những yêu cầu hiện ở thời điểm hiện tại, nhóm đề xuất một số hướng phát triển cho đề tài này.
\begin{itemize}
    \item Kết hợp với mô hình phân đoạn mạch máu trong gan thành một mô hình phân đoạn cả gan lẫn mạch máu hoàn chỉnh (vì mạch máu nằm trong gan nên kết hợp hai mô hình một lúc sẽ đem lại kết quả tốt hơn việc chạy riêng biệt hai mô hình và lấy 2 kết quả ghép lại).
    \item Xây dựng ứng dụng hiển thị kết quả phân đoạn và mô hình 3D cho hệ thống gan - mạch máu hoàn chỉnh.
    \item Xây dựng mô hình tương tự để phân đoạn những cơ quan khác trong cơ thể từ ảnh CT.
\end{itemize}




\bibliography{refs}{}
\bibliographystyle{plainurl}
%-	Danh mục TL tham khảo
%-	Phụ lục (nếu có)

\end{document}
